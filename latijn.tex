\documentclass[12pt,a4paper]{article}

\usepackage{amsmath,amsthm,amsfonts,amssymb}
\usepackage{booktabs}
\usepackage{multirow}
\usepackage{float}
\usepackage{graphicx}
\usepackage[dutch]{babel}
\usepackage{hyperref}

\linespread{1.3}

\title{Latijnse Morfologie}
\author{1LC}
\date{}

\begin{document}

\maketitle

\section{Naamwoorden}

\subsection{Inleiding}

\begin{table}[H]
\centering
\begin{tabular}{ll}
\toprule
\textsc{Benaming} & \textsc{Functies} \\
\midrule
nominatief & onderwerp en naamwoordelijk deel \\
vocatief   & aangesproken persoon \\
accusatief & lijdend voorwerp; na voorzetsels \\
genitief   & `van'-bepaling; bijvoeglijke bepaling \\
datief     & meewerkend voorwerp; na voorzetsels \\
ablatief   & bijwoordelijke bepaling; na voorzetsels \\
\bottomrule
\end{tabular}
\caption{Latijnse naamvallen}
\label{tab:naamval}
\end{table}

\paragraph{Vertaaltips}
\begin{itemize}
    \item datief vertaal je meestal met \emph{aan} of \emph{voor} (personen)
    \item ablatief vertaal je meestal met \emph{met} of \emph{door} (zaken)
\end{itemize}


\subsection{Verbuigingsmodellen}

\subsubsection{Naamwoorden van de eerste klasse}
\begin{itemize}
    \item mannelijke naamwoorden op \textbf{-us}, genitief enkelvoud op \textbf{-i}
    \item enkele mannelijke naamwoorden op \textbf{-er} en \textbf{-a}
    \item vrouwelijke naamwoorden op \textbf{-a}, genitief enkelvoud op \textbf{-ae}
    \item onzijdige naamwoorden op \textbf{-um}, genitief enkelvoud op \textbf{-i}
\end{itemize}

\begin{table}[H]
\centering
\begin{tabular}{ r | l l l | l l l }
\toprule
 & \multicolumn{3}{c|}{\textsc{enk}} & \multicolumn{3}{c}{\textsc{mv}} \\
 & \multicolumn{1}{c}{\textsc{m}} & \multicolumn{1}{c}{\textsc{v}} & \multicolumn{1}{c|}{\textsc{o}} & \multicolumn{1}{c}{\textsc{m}} & \multicolumn{1}{c}{\textsc{v}} & \multicolumn{1}{c}{\textsc{o}} \\ 
\midrule
\textsc{nom} & av-\emph{us} & ros-\emph{a}  & don-\emph{um} & av-\emph{i}    & ros-\emph{ae}   & don-\emph{a} \\
\textsc{voc} & av-\emph{e}  & ros-\emph{a}  & don-\emph{um} & av-\emph{i}    & ros-\emph{ae}   & don-\emph{a} \\
\textsc{acc} & av-\emph{um} & ros-\emph{am} & don-\emph{um} & av-\emph{os}   & ros-\emph{as}   & don-\emph{a} \\
\textsc{gen} & av-\emph{i}  & ros-\emph{ae} & don-\emph{i}  & av-\emph{orum} & ros-\emph{arum} & don-\emph{orum} \\
\textsc{dat} & av-\emph{o}  & ros-\emph{ae} & don-\emph{o}  & av-\emph{is}   & ros-\emph{is}   & don-\emph{is} \\
\textsc{abl} & av-\emph{o}  & ros-\emph{a} & don-\emph{o} & av-\emph{is} & ros-\emph{is} & don-\emph{is} \\
\bottomrule
\end{tabular}
\caption{Naamwoorden van de eerste klasse}
\label{tab:nw1}
\end{table}

\subsubsection{Substantieven van de tweede klasse}
\begin{itemize}
    \item genitief enkelvoud op \textbf{-is}
\end{itemize}

\begin{table}[H]
\centering
\begin{tabular}{ r | l l l | l l l }
\toprule
 & \multicolumn{3}{c|}{\textsc{enk}} & \multicolumn{3}{c}{\textsc{mv}} \\
 & \multicolumn{1}{c}{\textsc{m}} & \multicolumn{1}{c}{\textsc{v}} & \multicolumn{1}{c|}{\textsc{o}} & \multicolumn{1}{c}{\textsc{m}} & \multicolumn{1}{c}{\textsc{v}} & \multicolumn{1}{c}{\textsc{o}} \\ 
\midrule
\textsc{nom} & dux            & mater          & corpus           & duc-\emph{es}   & matr-\emph{es}   & corpor-\emph{a} \\
\textsc{acc} & duc-\emph{em}  & matr-\emph{em} & corpus           & duc-\emph{es}   & matr-\emph{es}   & corpor-\emph{a} \\
\textsc{gen} & duc-\emph{is}  & matr-\emph{is} & corpor-\emph{is} & duc-\emph{um}   & matr-\emph{um}   & corpor-\emph{um} \\
\textsc{dat} & duc-\emph{i}   & matr-\emph{i}  & corpor-\emph{i}  & duc-\emph{ibus} & matr-\emph{ibus} & corpor-\emph{ibus} \\
\textsc{abl} & duc-\emph{e}   & matr-\emph{e}  & corpor-\emph{e}  & duc-\emph{ibus} & matr-\emph{ibus} & corpor-\emph{ibus} \\
\bottomrule
\end{tabular}
\caption{Substantieven van de tweede klasse}
\label{tab:subst2}
\end{table}

\subsubsection{Adjectieven van de tweede klasse}

\begin{table}[H]
\centering
\begin{tabular}{ r | l l l | l l l }
\toprule
 & \multicolumn{3}{c|}{\textsc{enk}} & \multicolumn{3}{c}{\textsc{mv}} \\
 & \multicolumn{1}{c}{\textsc{m}} & \multicolumn{1}{c}{\textsc{v}} & \multicolumn{1}{c|}{\textsc{o}} & \multicolumn{1}{c}{\textsc{m}} & \multicolumn{1}{c}{\textsc{v}} & \multicolumn{1}{c}{\textsc{o}} \\ 
\midrule
\textsc{nom} & felix           & felix           & felix           & felic-\emph{es}   & felic-\emph{es}   & felic-\emph{ia} \\
\textsc{acc} & felic-\emph{em} & felic-\emph{em} & felix           & felic-\emph{es}   & felic-\emph{es}   & felic-\emph{ia} \\
\textsc{gen} & felic-\emph{is} & felic-\emph{is} & felic-\emph{is} & felic-\emph{ium}  & felic-\emph{ium}  & felic-\emph{ium} \\
\textsc{dat} & felic-\emph{i}  & felic-\emph{i}  & felic-\emph{i}  & felic-\emph{ibus} & felic-\emph{ibus} & felic-\emph{ibus} \\
\textsc{abl} & felic-\emph{i}  & felic-\emph{i}  & felic-\emph{i}  & felic-\emph{ibus} & felic-\emph{ibus} & felic-\emph{ibus} \\
\bottomrule
\end{tabular}
\caption{Adjectieven van de tweede klasse}
\label{tab:adj2}
\end{table}

\subsubsection{Vorming van de vocatief}
De \emph{vocatief} is altijd gelijk aan de nominatief, behalve in het mannelijk enkelvoud van de 1ste klasse:
\begin{itemize}
    \item woorden op \textbf{-us}  $\rightarrow$ \textbf{-e}
    \item woorden op \textbf{-ius} $\rightarrow$ \textbf{-i}
    \item woorden op \textbf{-eus} blijven onveranderd
\end{itemize}

\subsubsection{Afwijkingen bij voornaamwoorden van de tweede klasse}
\begin{itemize}
    \item Een aantal substantieven vertoont genitief meervoud op \textbf{-ium} \\
        civis, navis, collis, urbs, pars, fons
    \item Vier onzijdige substantieven volgen volledig de verbuiging van de adjectieven \\
        animal, milia, mare, moenia
    \item Vier adjectieven volgen volledig de verbuiging van de substantieven \\
        dives, vetus, pauper, princeps
\end{itemize}

\subsection{Voorzetsels}

\begin{table}[H]
\centering
\begin{tabular}{ c | c }
\toprule
\textsc{Voorzetsels + Acc} & \textsc{Voorzetsels + Abl} \\
\midrule
ante    & cum \\
per     & e(x) \\
apud    & sub \\
post    & sine \\
ad      & de \\
inter   & in \\
praeter & a(b) \\
in      & \\
ad      & \\
\bottomrule
\end{tabular}
\caption{Voorzetsels met specifieke naamval}
\label{tab:voorz}
\end{table}


\section{Voornaamwoorden}

\subsection{Persoonlijke voornaamwoorden}

\subsubsection{Reflexief}

\begin{table}[H]
\centering
\begin{tabular}{ r | l l l | l l l }
\toprule
 & \multicolumn{3}{c|}{\textsc{enk}} & \multicolumn{3}{c}{\textsc{mv}} \\
 & \multicolumn{1}{c}{\textsc{1ste p}} & \multicolumn{1}{c}{\textsc{2de p}} & \multicolumn{1}{c|}{\textsc{3de p}} & \multicolumn{1}{c}{\textsc{1ste p}} & \multicolumn{1}{c}{\textsc{2de p}} & \multicolumn{1}{c}{\textsc{3de p}} \\ 
\midrule
\textsc{nom} & ego  & tu   & -    & nos         & vos          & - \\
\textsc{acc} & me   & te   & se   & nos         & vos          & se \\
\textsc{gen} & mei  & tui  & sui  & nostr-i/-um & vestr-i/-um  & sui \\
\textsc{dat} & mihi & tibi & sibi & nobis       & vobis        & sibi \\
\textsc{abl} & me   & te   & se   & nobis       & vobis        & se \\
\bottomrule
\end{tabular}
\caption{Reflexieve persoonlijke voornaamwoorden}
\label{tab:reflvnw}
\end{table}

\subsubsection{Niet-reflexief}

\begin{table}[H]
\centering
\begin{tabular}{ r | l l l | l l l }
\toprule
 & \multicolumn{3}{c|}{\textsc{enk}} & \multicolumn{3}{c}{\textsc{mv}} \\
 & \multicolumn{1}{c}{\textsc{m}} & \multicolumn{1}{c}{\textsc{v}} & \multicolumn{1}{c|}{\textsc{o}} & \multicolumn{1}{c}{\textsc{m}} & \multicolumn{1}{c}{\textsc{v}} & \multicolumn{1}{c}{\textsc{o}} \\ 
\midrule
\textsc{nom} & is  & ea                     & id & ii    & eae       & ea \\
\textsc{acc} & eum & eam                    & id & eos   & eas       & ea \\
\textsc{gen} & eius   & eius                   & eius  & eorum & earum     & eorum \\
\textsc{dat} & ei   & ei                     & ei  & eis (iis)     & eis (iis) & eis (iis) \\
\textsc{abl} & eo  & e$\overline{\text{a}}$ & eo & eis (iis)     & eis (iis) & eis (iis) \\
\bottomrule
\end{tabular}
\caption{Niet-reflexieve voornaamwoorden}
\label{tab:ntreflvnw}
\end{table}

\subsubsection{Opmerkingen}

\begin{itemize}
    \item Het voorzetsel \emph{cum} wordt in de 1ste en 2de persoon achteraan het voornaamwoord gehecht: \\ \emph{mecum}, \emph{tecum}, \emph{nobiscum}, \emph{vobiscum}
\end{itemize}

\subsection{Aanwijzende voornaamwoorden}

\begin{table}[H]
\centering
\begin{tabular}{ r | l l l | l l l }
\toprule
 & \multicolumn{3}{c|}{\textsc{enk}} & \multicolumn{3}{c}{\textsc{mv}} \\
 & \multicolumn{1}{c}{\textsc{m}} & \multicolumn{1}{c}{\textsc{v}} & \multicolumn{1}{c|}{\textsc{o}} & \multicolumn{1}{c}{\textsc{m}} & \multicolumn{1}{c}{\textsc{v}} & \multicolumn{1}{c}{\textsc{o}} \\ 
\midrule
\textsc{nom} & hic   & haec  & hoc   & hi    & hae   & haec  \\
\textsc{acc} & hunc  & hanc  & hoc   & hos   & has   & haec  \\
\textsc{gen} & huius & huius & huius & horum & harum & horum \\
\textsc{dat} & huic  & hiuc  & huic  & his   & his   & his   \\
\textsc{abl} & hoc   & hac   & hoc   & his   & his   & his   \\
\bottomrule
\end{tabular}
\caption{hic, haec, hoc}
\label{tab:hic}
\end{table}

\begin{table}[H]
\centering
\begin{tabular}{ r | l l l | l l l }
\toprule
 & \multicolumn{3}{c|}{\textsc{enk}} & \multicolumn{3}{c}{\textsc{mv}} \\
 & \multicolumn{1}{c}{\textsc{m}} & \multicolumn{1}{c}{\textsc{v}} & \multicolumn{1}{c|}{\textsc{o}} & \multicolumn{1}{c}{\textsc{m}} & \multicolumn{1}{c}{\textsc{v}} & \multicolumn{1}{c}{\textsc{o}} \\ 
\midrule
\textsc{nom} & iste   & ista   & istud  & isti    & istae   & ista  \\
\textsc{acc} & istum  & istam  & istud  & istos   & istas   & ista  \\
\textsc{gen} & istius & istius & istius & istorum & istarum & istorum \\
\textsc{dat} & isti   & isti   & isti   & istis   & istis   & istis   \\
\textsc{abl} & isto   & ista   & isto   & istis   & istis   & istis   \\
\bottomrule
\end{tabular}
\caption{iste, ista, istud}
\label{tab:iste}
\end{table}

\begin{table}[H]
\centering
\begin{tabular}{ r | l l l | l l l }
\toprule
 & \multicolumn{3}{c|}{\textsc{enk}} & \multicolumn{3}{c}{\textsc{mv}} \\
 & \multicolumn{1}{c}{\textsc{m}} & \multicolumn{1}{c}{\textsc{v}} & \multicolumn{1}{c|}{\textsc{o}} & \multicolumn{1}{c}{\textsc{m}} & \multicolumn{1}{c}{\textsc{v}} & \multicolumn{1}{c}{\textsc{o}} \\ 
\midrule
\textsc{nom} & ille   & illa   & illud  & illi    & illae   & illa  \\
\textsc{acc} & illum  & illam  & illud  & illos   & illas   & illa  \\
\textsc{gen} & illius & illius & illius & illorum & illarum & illorum \\
\textsc{dat} & illi   & illi   & illi   & illis   & illis   & illis   \\
\textsc{abl} & illo   & illa   & illo   & illis   & illis   & illis   \\
\bottomrule
\end{tabular}
\caption{ille, illa, illud}
\label{tab:ille}
\end{table}

\begin{table}[H]
\centering
\begin{tabular}{ r | l l l | l l l }
\toprule
 & \multicolumn{3}{c|}{\textsc{enk}} & \multicolumn{3}{c}{\textsc{mv}} \\
 & \multicolumn{1}{c}{\textsc{m}} & \multicolumn{1}{c}{\textsc{v}} & \multicolumn{1}{c|}{\textsc{o}} & \multicolumn{1}{c}{\textsc{m}} & \multicolumn{1}{c}{\textsc{v}} & \multicolumn{1}{c}{\textsc{o}} \\ 
\midrule
\textsc{nom} & idem    & eadem   & idem    & iidem           & eaedem          & eadem  \\
\textsc{acc} & eundem  & eandem  & idem    & eosdem          & easdem          & eadem  \\
\textsc{gen} & eiusdem & eiusdem & eiusdem & eorundem        & earundem        & eorundem \\
\textsc{dat} & eidem   & eidem   & eidem   & eisdem (iisdem) & eisdem (iisdem) & eisdem (iisdem)   \\
\textsc{abl} & eodem   & eadem   & eodem   & eisdem (iisdem) & eisdem (iisdem) & eisdem (iisdem)   \\
\bottomrule
\end{tabular}
\caption{idem, eadem, idem}
\label{tab:idem}
\end{table}

\begin{table}[H]
\centering
\begin{tabular}{ r | l l l | l l l }
\toprule
 & \multicolumn{3}{c|}{\textsc{enk}} & \multicolumn{3}{c}{\textsc{mv}} \\
 & \multicolumn{1}{c}{\textsc{m}} & \multicolumn{1}{c}{\textsc{v}} & \multicolumn{1}{c|}{\textsc{o}} & \multicolumn{1}{c}{\textsc{m}} & \multicolumn{1}{c}{\textsc{v}} & \multicolumn{1}{c}{\textsc{o}} \\ 
\midrule
\textsc{nom} & ipse   & ipsa   & ipsum  & ipsi    & ipsae   & ipsa  \\
\textsc{acc} & ipsum  & ipsam  & ipsum  & ipsos   & ipsas   & ipsa  \\
\textsc{gen} & ipsius & ipsius & ipsius & ipsorum & ipsarum & ipsorum \\
\textsc{dat} & ipsi   & ipsi   & ipsi   & ipsis   & ipsis   & ipsis   \\
\textsc{abl} & ipso   & ipsa   & ipso   & ipsis   & ipsis   & ipsis   \\
\bottomrule
\end{tabular}
\caption{ipse, ipsa, ipsum}
\label{tab:ipse}
\end{table}

\section{Bijwoord}

Een bijwoord is meestal een bijwoordelijke bepaling bij een werkwoord. Het is onveranderlijk.

De vorming van een bijwoord op basis van een adjectief:
\begin{itemize}
    \item 1ste klasse: stam + \textbf{-e} \\
    Uitzonderingen: bene, aliter, multum
    \item 2de klasse: stam + \textbf{-iter}, \textbf{-ter}, \textbf{-er} \\
    Uitzonderingen: facile, triste
\end{itemize}

\section{Werkwoorden}

\subsection{Regelmatige werkwoorden}

\begin{table}[H]
\centering
\begin{tabular}{ c | l l l l l }
\toprule
 & \textsc{1ste verv} & \textsc{2de verv}  & \textsc{3de verv}  & \textsc{4de verv}  & \textsc{5de verv} \\
\midrule
\textsc{Inf} & ama-\emph{re} & mone-\emph{re} & teg-\emph{e}-\emph{re} & audi-\emph{re} & cape-\emph{re} \\
\midrule
\multirow{6}{*}{\rotatebox{90}{\textsc{Ind Praesens}}} & am-\emph{o} & mone-\emph{o} & teg-\emph{o} & audi-\emph{o} & capi-\emph{o} \\
 & ama-\emph{s}   & mone-\emph{s}   & teg-\emph{i}-\emph{s}   & audi-\emph{s}           & capi-\emph{s}   \\
 & ama-\emph{t}   & mone-\emph{t}   & teg-\emph{i}-\emph{t}   & audi-\emph{t}           & capi-\emph{t}   \\
 & ama-\emph{mus} & mone-\emph{mus} & teg-\emph{i}-\emph{mus} & audi-\emph{mus}         & capi-\emph{mus} \\
 & ama-\emph{tis} & mone-\emph{tis} & teg-\emph{i}-\emph{tis} & audi-\emph{tis}         & capi-\emph{tis} \\
 & ama-\emph{nt}  & mone-\emph{nt}  & teg-\emph{u}-\emph{nt}  & audi-\emph{u}-\emph{nt} & capi-\emph{u}-\emph{nt} \\
\midrule
\multirow{2}{*}{\textsc{Imp}} & ama           & mone           & teg-\emph{e}           & audi           & cape \\
                              & ama-\emph{te} & mone-\emph{te} & teg-\emph{i}-\emph{te} & audi-\emph{te} & capi-\emph{te} \\
\bottomrule
\end{tabular}
\caption{Regelmatige werkwoorden (actief)}
\label{tab:regwwact}
\end{table}

\begin{table}[H]
\centering
\begin{tabular}{ c | l l l l l }
\toprule
 & \textsc{1ste verv} & \textsc{2de verv}  & \textsc{3de verv}  & \textsc{4de verv}  & \textsc{5de verv} \\
\midrule
\textsc{Inf} & ama-\emph{ri} & mone-\emph{ri} & teg-\emph{i} & audi-\emph{ri} & capi \\
\midrule
\multirow{6}{*}{\rotatebox{90}{\textsc{Ind Praesens}}} & am-\emph{or} & mone-\emph{or} & teg-\emph{or} & audi-\emph{or} & capi-\emph{or} \\
 & ama-\emph{ris}  & mone-\emph{ris}  & teg-\emph{e}-\emph{ris}  & audi-\emph{ris}            & capi-\emph{ris}   \\
 & ama-\emph{tur}  & mone-\emph{tur}  & teg-\emph{i}-\emph{tur}  & audi-\emph{tur}            & capi-\emph{tur}   \\
 & ama-\emph{mur}  & mone-\emph{mur}  & teg-\emph{i}-\emph{mur}  & audi-\emph{mur}            & capi-\emph{mur} \\
 & ama-\emph{mini} & mone-\emph{mini} & teg-\emph{i}-\emph{mini} & audi-\emph{mini}           & capi-\emph{mini} \\
 & ama-\emph{ntur} & mone-\emph{ntur} & teg-\emph{u}-\emph{ntur} & audi-\emph{u}-\emph{ntur} & capi-\emph{u}-\emph{ntur} \\
\bottomrule
\end{tabular}
\caption{Regelmatige werkwoorden (passief)}
\label{tab:regwwpas}
\end{table}

\subsubsection{Uitzonderingen}
\begin{itemize}
    \item de imperatief enkelvoud van facere is \emph{fac}
    \item de imperatief enkelvoud van dicere is \emph{dic}
    \item de imperatief enkelvoud van ducere is \emph{duc}
\end{itemize}

\subsection{Eerste vervoeging: amare}

\begin{table}[H]
\centering
\begin{tabular}{ c | c | l l }
\toprule
 & & \multicolumn{2}{c}{\textsc{Amare}} \\
 & & \textsc{Actief} & \textsc{Passief} \\
\midrule
\textsc{Infinitief} & \textsc{Praesens} & ama-\emph{re} & ama-\emph{ri} \\
\midrule
\multirow{18}{*}{\rotatebox{90}{\textsc{Indicatief}}} & \multirow{6}{*}{\rotatebox{90}{\textsc{Praesens}}} & am-\emph{o} & am-\emph{or} \\
 & & ama-\emph{s}   & ama-\emph{ris} \\
 & & ama-\emph{t}   & ama-\emph{tur} \\
 & & ama-\emph{mus} & ama-\emph{mur} \\
 & & ama-\emph{tis} & ama-\emph{mini} \\
 & & ama-\emph{nt}  & ama-\emph{ntur} \\
\cmidrule{2-4}
 & \multirow{6}{*}{\rotatebox{90}{\textsc{Imperfectum}}} & ama-\emph{ba}-m & ama-\emph{ba}-r \\
 & & ama-\emph{ba}-s   & ama-\emph{ba}-ris \\
 & & ama-\emph{ba}-t   & ama-\emph{ba}-tur \\
 & & ama-\emph{ba}-mus & ama-\emph{ba}-mur \\
 & & ama-\emph{ba}-tis & ama-\emph{ba}-mini \\
 & & ama-\emph{ba}-nt  & ama-\emph{ba}-ntur \\
\cmidrule{2-4}
 & \multirow{6}{*}{\rotatebox{90}{\textsc{Futurum Simplex}}} & ama-\emph{b}-o & ama-\emph{b}-or \\
 & & ama-\emph{b}-\emph{i}-s   & ama-\emph{b}-\emph{e}-ris \\
 & & ama-\emph{b}-\emph{i}-t   & ama-\emph{b}-\emph{i}-tur \\
 & & ama-\emph{b}-\emph{i}-mus & ama-\emph{b}-\emph{i}-mur \\
 & & ama-\emph{b}-\emph{i}-tis & ama-\emph{b}-\emph{i}-mini \\
 & & ama-\emph{b}-\emph{u}-nt  & ama-\emph{b}-\emph{u}-ntur \\
\midrule
\multirow{2}{*}{\textsc{Imperatief}} & & ama & \\
 & & ama-\emph{te} & \\
\bottomrule
\end{tabular}
\caption{Eerste vervoeging}
\label{tab:amare}
\end{table}

\subsection{Tweede vervoeging: monere}

\begin{table}[H]
\centering
\begin{tabular}{ c | c | l l }
\toprule
\multicolumn{4}{c}{\textsc{Monere}} \\
\midrule
 & & \textsc{Actief} & \textsc{Passief} \\
\midrule
\textsc{Infinitief} & \textsc{Praesens} & ama-\emph{re} & ama-\emph{ri} \\
\midrule
\multirow{18}{*}{\rotatebox{90}{\textsc{Indicatief}}} & \multirow{6}{*}{\rotatebox{90}{\textsc{Praesens}}} & mone-\emph{o} & mone-\emph{or} \\
 & & mone-\emph{s}   & mone-\emph{ris} \\
 & & mone-\emph{t}   & mone-\emph{tur} \\
 & & mone-\emph{mus} & mone-\emph{mur} \\
 & & mone-\emph{tis} & mone-\emph{mini} \\
 & & mone-\emph{nt}  & mone-\emph{ntur} \\
\cmidrule{2-4}
 & \multirow{6}{*}{\rotatebox{90}{\textsc{Imperfectum}}} & mone-\emph{ba}-m & mone-\emph{ba}-r \\
 & & mone-\emph{ba}-s   & mone-\emph{ba}-ris \\
 & & mone-\emph{ba}-t   & mone-\emph{ba}-tur \\
 & & mone-\emph{ba}-mus & mone-\emph{ba}-mur \\
 & & mone-\emph{ba}-tis & mone-\emph{ba}-mini \\
 & & mone-\emph{ba}-nt  & mone-\emph{ba}-ntur \\
\cmidrule{2-4}
 & \multirow{6}{*}{\rotatebox{90}{\textsc{Futurum Simplex}}} & mone-\emph{b}-o & mone-\emph{b}-or \\
 & & mone-\emph{b}-\emph{i}-s   & mone-\emph{b}-\emph{e}-ris \\
 & & mone-\emph{b}-\emph{i}-t   & mone-\emph{b}-\emph{i}-tur \\
 & & mone-\emph{b}-\emph{i}-mus & mone-\emph{b}-\emph{i}-mur \\
 & & mone-\emph{b}-\emph{i}-tis & mone-\emph{b}-\emph{i}-mini \\
 & & mone-\emph{b}-\emph{u}-nt  & mone-\emph{b}-\emph{u}-ntur \\
\midrule
\multirow{2}{*}{\textsc{Imperatief}} & & mone & \\
 & & mone-\emph{te} & \\
\bottomrule
\end{tabular}
\caption{Tweede vervoeging}
\label{tab:monere}
\end{table}

\subsection{Derde vervoeging: tegere}

\begin{table}[H]
\centering
\begin{tabular}{ c | c | l l }
\toprule
\multicolumn{4}{c}{\textsc{Tegere}} \\
\midrule
 & & \textsc{Actief} & \textsc{Passief} \\
\midrule
\textsc{Infinitief} & \textsc{Praesens} & teg-\emph{e}-\emph{re} & teg-\emph{i} \\
\midrule
\multirow{18}{*}{\rotatebox{90}{\textsc{Indicatief}}} & \multirow{6}{*}{\rotatebox{90}{\textsc{Praesens}}} & teg-\emph{o} & teg-\emph{or} \\
 & & teg-\emph{i}-\emph{s}   & teg-\emph{e}-\emph{ris} \\
 & & teg-\emph{i}-\emph{t}   & teg-\emph{i}-\emph{tur} \\
 & & teg-\emph{i}-\emph{mus} & teg-\emph{i}-\emph{mur} \\
 & & teg-\emph{i}-\emph{tis} & teg-\emph{i}-\emph{mini} \\
 & & teg-\emph{u}-\emph{nt}  & teg-\emph{u}-\emph{ntur} \\
\cmidrule{2-4}
 & \multirow{6}{*}{\rotatebox{90}{\textsc{Imperfectum}}} & teg-\emph{eba}-m & teg-\emph{eba}-r \\
 & & teg-\emph{eba}-s   & teg-\emph{eba}-ris \\
 & & teg-\emph{eba}-t   & teg-\emph{eba}-tur \\
 & & teg-\emph{eba}-mus & teg-\emph{eba}-mur \\
 & & teg-\emph{eba}-tis & teg-\emph{eba}-mini \\
 & & teg-\emph{eba}-nt  & teg-\emph{eba}-ntur \\
\cmidrule{2-4}
 & \multirow{6}{*}{\rotatebox{90}{\textsc{Futurum Simplex}}} & teg-\emph{a}-m & teg-\emph{a}-r \\
 & & teg-\emph{e}-s   & teg-\emph{e}-ris \\
 & & teg-\emph{e}-t   & teg-\emph{e}-tur \\
 & & teg-\emph{e}-mus & teg-\emph{e}-mur \\
 & & teg-\emph{e}-tis & teg-\emph{e}-mini \\
 & & teg-\emph{e}-nt  & teg-\emph{e}-ntur \\
\midrule
\multirow{2}{*}{\textsc{Imperatief}} & & teg-\emph{e} & \\
 & & teg-\emph{i}-\emph{te} & \\
\bottomrule
\end{tabular}
\caption{Derde vervoeging}
\label{tab:tegere}
\end{table}

\subsection{Vierde vervoeging: audire}

\begin{table}[H]
\centering
\begin{tabular}{ c | c | l l }
\toprule
\multicolumn{4}{c}{\textsc{Audire}} \\
\midrule
 & & \textsc{Actief} & \textsc{Passief} \\
\midrule
\textsc{Infinitief} & \textsc{Praesens} & audi-\emph{re} & audi-\emph{ri} \\
\midrule
\multirow{18}{*}{\rotatebox{90}{\textsc{Indicatief}}} & \multirow{6}{*}{\rotatebox{90}{\textsc{Praesens}}} & audi-\emph{o} & audi-\emph{or} \\
 & & audi-\emph{s}   & audi-\emph{ris} \\
 & & audi-\emph{t}   & audi-\emph{tur} \\
 & & audi-\emph{mus} & audi-\emph{mur} \\
 & & audi-\emph{tis} & audi-\emph{mini} \\
 & & audi-\emph{u}-\emph{nt}  & audi-\emph{u}-\emph{ntur} \\
\cmidrule{2-4}
 & \multirow{6}{*}{\rotatebox{90}{\textsc{Imperfectum}}} & audi-\emph{eba}-m & audi-\emph{eba}-r \\
 & & audi-\emph{eba}-s   & audi-\emph{eba}-ris \\
 & & audi-\emph{eba}-t   & audi-\emph{eba}-tur \\
 & & audi-\emph{eba}-mus & audi-\emph{eba}-mur \\
 & & audi-\emph{eba}-tis & audi-\emph{eba}-mini \\
 & & audi-\emph{eba}-nt  & audi-\emph{eba}-ntur \\
\cmidrule{2-4}
 & \multirow{6}{*}{\rotatebox{90}{\textsc{Futurum Simplex}}} & audi-\emph{a}-m & audi-\emph{a}-r \\
 & & audi-\emph{e}-s   & audi-\emph{e}-ris \\
 & & audi-\emph{e}-t   & audi-\emph{e}-tur \\
 & & audi-\emph{e}-mus & audi-\emph{e}-mur \\
 & & audi-\emph{e}-tis & audi-\emph{e}-mini \\
 & & audi-\emph{e}-nt  & audi-\emph{e}-ntur \\
\midrule
\multirow{2}{*}{\textsc{Imperatief}} & & audi & \\
 & & audi-\emph{te} & \\
\bottomrule
\end{tabular}
\caption{Vierde vervoeging}
\label{tab:audire}
\end{table}

\subsection{Vijfde vervoeging: capere}

\begin{table}[H]
\centering
\begin{tabular}{ c | c | l l }
\toprule
\multicolumn{4}{c}{\textsc{Capere}} \\
\midrule
 & & \textsc{Actief} & \textsc{Passief} \\
\midrule
\textsc{Infinitief} & \textsc{Praesens} & cape-\emph{re} & capi \\
\midrule
\multirow{18}{*}{\rotatebox{90}{\textsc{Indicatief}}} & \multirow{6}{*}{\rotatebox{90}{\textsc{Praesens}}} & capi-\emph{o} & capi-\emph{or} \\
 & & capi-\emph{s}   & capi-\emph{ris} \\
 & & capi-\emph{t}   & capi-\emph{tur} \\
 & & capi-\emph{mus} & capi-\emph{mur} \\
 & & capi-\emph{tis} & capi-\emph{mini} \\
 & & capi-\emph{u}-\emph{nt}  & capi-\emph{u}-\emph{ntur} \\
\cmidrule{2-4}
 & \multirow{6}{*}{\rotatebox{90}{\textsc{Imperfectum}}} & capi-\emph{eba}-m & capi-\emph{eba}-r \\
 & & capi-\emph{eba}-s   & capi-\emph{eba}-ris \\
 & & capi-\emph{eba}-t   & capi-\emph{eba}-tur \\
 & & capi-\emph{eba}-mus & capi-\emph{eba}-mur \\
 & & capi-\emph{eba}-tis & capi-\emph{eba}-mini \\
 & & capi-\emph{eba}-nt  & capi-\emph{eba}-ntur \\
\cmidrule{2-4}
 & \multirow{6}{*}{\rotatebox{90}{\textsc{Futurum Simplex}}} & capi-\emph{a}-m & capi-\emph{a}-r \\
 & & capi-\emph{e}-s   & capi-\emph{e}-ris \\
 & & capi-\emph{e}-t   & capi-\emph{e}-tur \\
 & & capi-\emph{e}-mus & capi-\emph{e}-mur \\
 & & capi-\emph{e}-tis & capi-\emph{e}-mini \\
 & & capi-\emph{e}-nt  & capi-\emph{e}-ntur \\
\midrule
\multirow{2}{*}{\textsc{Imperatief}} & & cape & \\
 & & capi-\emph{te} & \\
\bottomrule
\end{tabular}
\caption{Vijfde vervoeging}
\label{tab:capere}
\end{table}

\subsection{Onregelmatige werkwoorden}

\begin{table}[H]
\centering
\begin{tabular}{ c | c | l l l l }
\toprule
\multicolumn{6}{c}{\textsc{Onregelmatige werkwoorden}} \\
\midrule
\textsc{Inf} & \textsc{Pr} & esse & pos-se & i-re & fi-e-ri \\
\midrule
\multirow{18}{*}{\rotatebox{90}{\textsc{Indicatief}}} & \multirow{6}{*}{\rotatebox{90}{\textsc{Praesens}}} & sum & pos-sum & e-o & fi-o\\
 & & es    & pot-es    & i-s    & fi-s    \\
 & & est   & pot-est   & i-t    & fi-t    \\
 & & sumus & pos-sumus & i-mus  & fi-mus  \\
 & & estis & pot-estis & i-tis  & fi-tis  \\
 & & sunt  & pos-sunt  & e-u-nt & fi-u-nt \\
\cmidrule{2-6}
 & \multirow{6}{*}{\rotatebox{90}{\textsc{Imperfectum}}} & \emph{era}-m & pot-\emph{era}-m & i-\emph{ba}-m & fi-\emph{eba}-m \\
 & & \emph{era}-s   & pot-\emph{era}-s   & i-\emph{ba}-s   & fi-\emph{eba}-s \\
 & & \emph{era}-t   & pot-\emph{era}-t   & i-\emph{ba}-t   & fi-\emph{eba}-t \\
 & & \emph{era}-mus & pot-\emph{era}-mus & i-\emph{ba}-mus & fi-\emph{eba}-mus \\
 & & \emph{era}-tis & pot-\emph{era}-tis & i-\emph{ba}-tis & fi-\emph{eba}-tis \\
 & & \emph{era}-nt  & pot-\emph{era}-nt  & i-\emph{ba}-nt  & fi-\emph{eba}-nt \\
\cmidrule{2-6}
 & \multirow{6}{*}{\rotatebox{90}{\textsc{Futurum Simplex}}} & \emph{er}-o & pot-\emph{er}-o & i-\emph{b}-o & fi-\emph{a}-m \\
 & & \emph{eri}-s   & pot-\emph{eri}-s   & i-\emph{b}-\emph{i}-s   & fi-\emph{e}-s \\
 & & \emph{eri}-t   & pot-\emph{eri}-t   & i-\emph{b}-\emph{i}-t   & fi-\emph{e}-t \\
 & & \emph{eri}-mus & pot-\emph{eri}-mus & i-\emph{b}-\emph{i}-mus & fi-\emph{e}-mus \\
 & & \emph{eri}-tis & pot-\emph{eri}-tis & i-\emph{b}-\emph{i}-tis & fi-\emph{e}-tis \\
 & & \emph{eru}-nt  & pot-\emph{eru}-nt  & i-\emph{b}-\emph{u}-nt  & fi-\emph{e}-nt \\
\midrule
\multirow{2}{*}{\textsc{Imp}} & & es           & & i           & \\
                              & & es-\emph{te} & & i-\emph{te} & \\
\bottomrule
\end{tabular}
\caption{Onregelmatige werkwoorden: esse, posse, ire, fieri}
\label{tab:onregww1}
\end{table}

\begin{table}[H]
\centering
\begin{tabular}{ c | c | l l l }
\toprule
\multicolumn{5}{c}{\textsc{Onregelmatige werkwoorden}} \\
\midrule
\textsc{Inf} & \textsc{Pr} & velle & nolle & malle \\
\midrule
\multirow{18}{*}{\rotatebox{90}{\textsc{Indicatief}}} & \multirow{6}{*}{\rotatebox{90}{\textsc{Praesens}}} & vol & nolo & malo \\
 & & vis     & non vis    & mavis    \\
 & & vult    & non vult   & mavult   \\
 & & volumus & nolumus    & malumus  \\
 & & vultis  & non vultis & mavultis \\
 & & volunt  & nolunt     & malunt   \\
\cmidrule{2-5}
 & \multirow{6}{*}{\rotatebox{90}{\textsc{Imperfectum}}} & vol-\emph{eba}-m & nol-\emph{era}-m & mal-\emph{eba}-m \\
 & & vol-\emph{eba}-s   & nol-\emph{eba}-s   & mal-\emph{eba}-s   \\
 & & vol-\emph{eba}-t   & nol-\emph{eba}-t   & mal-\emph{eba}-t   \\
 & & vol-\emph{eba}-mus & nol-\emph{eba}-mus & mal-\emph{eba}-mus \\
 & & vol-\emph{eba}-tis & nol-\emph{eba}-tis & mal-\emph{eba}-tis \\
 & & vol-\emph{eba}-nt  & nol-\emph{eba}-nt  & mal-\emph{eba}-nt  \\
\cmidrule{2-5}
 & \multirow{6}{*}{\rotatebox{90}{\textsc{Futurum Simplex}}} & vol-\emph{a}-m & nol-\emph{a}-m & mal-\emph{a}-m \\
 & & vol-\emph{e}-s   & nol-\emph{e}-s   & mal-\emph{e}-s   \\
 & & vol-\emph{e}-t   & nol-\emph{e}-t   & mal-\emph{e}-t   \\
 & & vol-\emph{e}-mus & nol-\emph{e}-mus & mal-\emph{e}-mus \\
 & & vol-\emph{e}-tis & nol-\emph{e}-tis & mal-\emph{e}-tis \\
 & & vol-\emph{e}-nt  & nol-\emph{e}-nt  & mal-\emph{e}-nt  \\
\midrule
\multirow{2}{*}{\textsc{Imp}} & & & noli           & \\
                              & & & noli-\emph{te} & \\
\bottomrule
\end{tabular}
\caption{Onregelmatige werkwoorden: velle, nolle, malle}
\label{tab:onregww2}
\end{table}

\begin{table}[H]
\centering
\begin{tabular}{ c | c | l l }
\toprule
\multicolumn{4}{c}{\textsc{Ferre}} \\
\midrule
 & & \textsc{Actief} & \textsc{Passief} \\
\midrule
\textsc{Inf} & \textsc{Pr} & fer-\emph{re} & fer-\emph{ri} \\
\midrule
\multirow{18}{*}{\rotatebox{90}{\textsc{Indicatief}}} & \multirow{6}{*}{\rotatebox{90}{\textsc{Praesens}}} & fer-\emph{o} & fer-\emph{or} \\
 & & fer-\emph{s}            & fer-\emph{ris} \\
 & & fer-\emph{t}            & fer-\emph{tur} \\
 & & fer-\emph{i}-\emph{mus} & fer-\emph{i}-\emph{mur} \\
 & & fer-\emph{tis}          & fer-\emph{i}-\emph{mini} \\
 & & fer-\emph{u}-\emph{nt}  & fer-\emph{u}-\emph{ntur} \\
\cmidrule{2-4}
 & \multirow{6}{*}{\rotatebox{90}{\textsc{Imperfectum}}} & fer-\emph{eba}-m & fer-\emph{eba}-r \\
 & & fer-\emph{eba}-s   & fer-\emph{eba}-ris \\
 & & fer-\emph{eba}-t   & fer-\emph{eba}-tur \\
 & & fer-\emph{eba}-mus & fer-\emph{eba}-mur \\
 & & fer-\emph{eba}-tis & fer-\emph{eba}-mini \\
 & & fer-\emph{eba}-nt  & fer-\emph{eba}-ntur \\
\cmidrule{2-4}
 & \multirow{6}{*}{\rotatebox{90}{\textsc{Futurum Simplex}}} & fer-\emph{a}-m & fer-\emph{a}-r \\
 & & fer-\emph{e}-s   & fer-\emph{e}-ris \\
 & & fer-\emph{e}-t   & fer-\emph{e}-tur \\
 & & fer-\emph{e}-mus & fer-\emph{e}-mur \\
 & & fer-\emph{e}-tis & fer-\emph{e}-mini \\
 & & fer-\emph{e}-nt  & fer-\emph{e}-ntur \\
\midrule
\multirow{2}{*}{\textsc{Imp}} & & fer & \\
 & & fer-\emph{te} & \\
\bottomrule
\end{tabular}
\caption{Onregelmatige werkwoorden: ferre}
\label{tab:ferre}
\end{table}

\end{document}